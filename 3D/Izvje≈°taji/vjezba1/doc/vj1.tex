\documentclass{article}

\usepackage[T1]{fontenc}
\usepackage[utf8]{inputenc}
\usepackage{a4wide}
\usepackage{xcolor}
\usepackage{listings}
%\usepackage{listingsutf8}
%\usepackage{lmodern}
\usepackage[croatian]{babel}
\usepackage{titlesec}
\usepackage{titletoc}
\usepackage{graphicx}
\usepackage{caption}
\usepackage[tikz]{mdframed}
\usepackage{subcaption}
\usepackage{AnonymousPro}

\renewcommand{\rmdefault}{cmr}
\renewcommand{\ttdefault}{AnonymousPro}
\renewcommand{\figurename}{Slika}
\setlength{\fboxsep}{0pt}
\setlength{\parindent}{0pt}
\definecolor{code}{RGB}{255,255,200}
\definecolor{console}{RGB}{225,225,225}
\definecolor{comment}{RGB}{100,100,100}
%\DeclareTextFontCommand{\emph}{\bfseries\em}

\newcommand{\localtextbulletone}{\textcolor{black}{\raisebox{.45ex}{\rule{.6ex}{.6ex}}}}

\DeclareCaptionLabelFormat{listing}{Prijepis \thesection.#2.}
\DeclareCaptionLabelSeparator{listing}{ }
\DeclareCaptionFont{white}{\bf\sffamily\color{white}}
\DeclareCaptionFont{codesection}{\bf\ttfamily\color{white}}
\DeclareCaptionLabelFormat{continued}{#1#2}
\DeclareCaptionFormat{codesection}{%
	\colorbox{gray}{%
		\fbox{%
		\begin{minipage}{0.9158\textwidth}
			\vspace{4pt}
			\flushright
				#3
			\vspace{4pt}
		\end{minipage}
	}}}
\DeclareCaptionStyle{codesection}[]{%
		format=codesection,
		labelseparator=listing,
		%labelformat=listing,
		%labelfont=white,
		textfont=codesection,
		margin=15.5pt,
	}
\DeclareCaptionFormat{listing}{%
	\colorbox{gray}{%
		\fbox{%
		\begin{minipage}{0.9158\textwidth}
			\vspace{4pt}
			\hspace{4pt}
				#1#2#3
			\vspace{4pt}
		\end{minipage}
	}}}
\DeclareCaptionStyle{listing}[]{%
		format=listing,
		labelseparator=listing,
		labelformat=listing,
		labelfont=white,
		textfont=white,
		margin=15.5pt,
	}
\newcommand{\setlstcaption}[1]{%
	\captionsetup[lstlisting]{
		style=#1,
		justification=raggedright,
		singlelinecheck=off,
		}
}

\lstdefinestyle{default}{%
	basicstyle=\small\ttfamily\linespread{1.1},
	breaklines=true,
	inputencoding=ansinew,
	columns=fullflexible,
	rulecolor=\color{black},
	xleftmargin=25pt,
	xrightmargin=25.5pt,
	frame=single,
	framesep=9pt,
	belowcaptionskip=0pt,
	showstringspaces=false,
		literate={ć}{{\'{c}}}1{č}{{\v{c}}}1{đ}{{\dj{}}}1{š}{{\v{s}}}1{ž}{{\v{z}}}1{Ć}{{\'{C}}}1{Č}{{\v{C}}}1{Š}{{\v{S}}}1{Ž}{{\v{Z}}}1,
}
\lstdefinestyle{code}{%
	style=default,
	stringstyle=\it\ttfamily,
	commentstyle=\it\ttfamily\color{comment},
	numbers=left,
	numbersep=1.5em,
	numberstyle=\bf\small\ttfamily\color{darkgray},
	mathescape,
	backgroundcolor=\color{code},
	firstnumber=\lst@firstline,
}
\lstdefinestyle{stdout}{%
	style=default,
	backgroundcolor=\color{console},
}
\lstnewenvironment{stdout}[2]{%
	\lstset{%
		style=stdout,
		caption={#1},
		label=#2,
		}}%
	{}
\lstdefinestyle{C++}{%
	style=code,
	language=C++,
}

\newenvironment{codelisting}{%
	\setlstcaption{listing}
	\vspace{1em}
}
{\vspace{1em}}

\newenvironment{codesection}{%
	\setlstcaption{codesection}
	\addtocounter{lstlisting}{-1}
	\vspace{1em}
}
{\vspace{1em}}


\newcommand{\reflst}[2]{%
	\ref{sec:#1}.\ref{lst:#2}.
}

\newcommand{\QandA}[2]{
	\vspace{2.5em}
	\localtextbulletone \hspace{2pt} {\bf #1}

	\vspace{1em}
	#2}

\titlelabel{\thetitle.\quad}
\titleformat{\section}
{\huge\bfseries\large}
{\thesection.}
{0.5 em}
{}

\titleformat{\subsection}
{\bfseries}
{\thesubsection}
{0.5em}
{}

\newcommand{\tottf}[1]{%
	{\fontfamily{AnonymousPro}\selectfont#1}}

\newcommand{\codedir}{C:/Users/Jakov/FESB/2. semestar/3D simulacije/Izvještaji/vjezba1}

\title{Vježba 1}
\author{Jakov Spahija}
\begin{document}
\maketitle
\vspace{10em}
\tableofcontents
\pagebreak
\vspace{2em}


%%%%%%%%%%%%%%%%%%%%%%%%%%%%%%%%%%%%%%%%%%%%%%%%%%%%%%%%%%%%%%%%%%%%%%%%%%
\section{Tipovi i objekti}
\label{sec:tio}
\setcounter{lstlisting}{0}

U uvodnoj vježbi, izvode se objektno orijentirane mogućnosti C++. Deklaracija te overload-nje funckija, klasa.


U header datoteci \tottf{Circle.h} definiran je objekt \tottf{Circle} koji se sastoji of:
\begin{itemize}
	\item član {\bf{\tottf{double}}} \tottf{\_radius}
	\item konstruktorom koji postavlja jedinstvenu varijablu tipa {\bf{\tottf{double}}} na \tottf{1}.\tottf{0}
	\item konstantna, virtualna member funckija \tottf{Area} koja vraća {\bf\tottf{double}}
	\item konstantna, virtualna member funckija \tottf{ToString} koja vraća {\bf\tottf{wstring}}
	\item virtualni destruktor 
\end{itemize}

\begin{codelisting}
	\lstinputlisting[style=C++,caption=Circle.h,label=lst:circleh]{\codedir/Types and Objects/Circle.h}
\end{codelisting}

Također su deklarirane i 3 funkcije \tottf{f}, \tottf{g} i \tottf{h}.

Definicija virtualnih funkcija klase, te ostale funkcije unutar \tottf{{\bf{namespace}} abc}, se odvija kasnije u datoteci \tottf{Circle.cpp} Prijepis \reflst{tio}{circlecpp}.

\pagebreak

\begin{codelisting}
\lstinputlisting[style=C++,firstline=10,caption=Circle.cpp,label=lst:circlecpp]{\codedir/Types and Objects/Circle.cpp}
\end{codelisting}

{\bf What would happen if you were to create a non-dynamic object inside the function and return its address?}

\vspace{1em}

Životni vijek podataka deklariranih na stogu unutar nekog scope-a, traje za sve izraze unutar tog scope-a. Takav objekt će biti uništen/izbrisan.

\pagebreak

\begin{codesection}
\lstinputlisting[style=stdout,caption=Terminal]{\codedir/Types and Objects/terminal.std}
\end{codesection}

\pagebreak

%%%%%%%%%%%%%%%%%%%%%%%%%%%%%%%%%%%%%%%%%%%%%%%%%%%%%%%%%%%%%%%%%%%%%%%%%%
\section{Reference}
\label{sec:ref}
\setcounter{lstlisting}{0}

Izraz u C++ se karaterizira sa tipom i kategorijom vrijednosti. \emph{Rvalue}, \emph{lvalue} su nam najkorisnije kategorije za promatranje, dok su \emph{xvalue}(e{\bf x}piring), \emph{glvalue}({\bf g}lobal {\bf l}eft), \emph{prvalue}({\bf p}ure {\bf r}ight) primitivne kategorije.

\begin{itemize}
	\item \emph{xvalue} je tranzicijska kategorija vrijednosti, koja služi u produživanju životnog vijeka resursa. \tottf{std::move} vraća takav izraz.
	\item \emph{lvalue}/\emph{rvalue} je \emph{glvalue}/\emph{prvalue} koji može biti i \emph{xvalue}, respektivno
\end{itemize}


\QandA%
{What would happen if you were to increment the reference’s value?}%
{Lijeva referenca je u ovom slučaju samo referenca koji nije promjenjiva.}
\begin{codesection}
\lstinputlisting[style=C++,caption=Something.h,firstline=19,lastline=22]{\codedir/References/Something.h}
\end{codesection}

\QandA%
{What would happen if you were to add another pointer to \tottf{{\bf int}} and assign to it the address of expression 100? And if the pointer was a read-only pointer?}%
{Takav izraz nije \emph{lvalue}, a ako je izraz read-only pokazivač onda se može pridjeliti}

\begin{codesection}
\lstinputlisting[style=C++,caption=Program.cpp,firstline=16,lastline=16]{\codedir/References/Program.cpp}
\end{codesection}


\QandA%
{What would happen if you tried to assign something else to l?}%
{Efektivno se pridjeli ta vrijednost, onoj lvalue na koju se l referencira, u ovom slučaju \tottf{v}}

\QandA%
{What would happen if you tried assigning v to a constant l-value reference to int?}%
{Ako pridjelimo \tottf{v} nekakvoj \tottf{\textbf{const}} \tottf{int\&}, preko takve se reference neće moći mijenjati.}

\pagebreak

\QandA%
{Increment l and output v’s value to the console. Can you assign 100 to an l-value reference to int?}%
\begin{codesection}
\lstinputlisting[style=C++,caption=Program.cpp,firstline=40,lastline=41]{\codedir/References/Program.cpp}
\end{codesection}

\QandA%
{Add a constant l-value reference cl to 100. Can you increment cl?}%
{\tottf{cl} nije definirana kao referenca preko koje možemo modificirati.}

\QandA%
{What would have happened had you uncommented the overloaded function prior to making calls to h?}%
{Tijekom \emph{compile time}, kompajler odredi koju funkciju poziva. Pod \emph{overload resolution} kriterijima, spadaju i kategorije vrijednosti parametara. Ako je na jedini parametar proslijeđen \emph{lvalue} izraz onda se preferira funkcija sa definicijom parametra \emph{lvalue} reference(\tottf{\textbf{const} T\&}). Također vrijedi isto i za \emph{rvalue} izraz te referencu(\tottf{T\&\&}).}
\begin{codesection}
\lstinputlisting[style=C++,caption=Something.h,firstline=29,lastline=36]{\codedir/References/Something.h}
\end{codesection}

\pagebreak

\begin{codesection}
\lstinputlisting[style=stdout,caption=Terminal]{\codedir/References/terminal.std}
\end{codesection}

\pagebreak

%%%%%%%%%%%%%%%%%%%%%%%%%%%%%%%%%%%%%%%%%%%%%%%%%%%%%%%%%%%%%%%%%%%%%%%%%%%
\section{Semantika}
\label{sec:sem}
\setcounter{lstlisting}{0}

\emph{Copy} i \emph{Move} konstruktori (CC, MC), su inače implicitno definirani od strane kompajlera. Oni se pozivaju kad se objekt inicijalizira.
\emph{Assignment} semantika specificra kako će se objektu, koji je već inicijaliziran, pridjeliti neka vrijednost.\\

Ovakva konfiguracija konstruktora je postavljena da bi se, kad je predana kao argument, \emph{rvalue} referenca, primjeni \emph{move} semanitka, koja efektivno \textit{'krade'} resurse of objekta na koji referenca pokazuje.
Inače, \emph{copy} semantika se primjenjuje u slučaju \emph{lvalue} reference.

\emph{Overload resolution} odabire poziv funkcija na takav način.

\begin{codesection}
	\lstinputlisting[style=C++,caption=Something.h,firstline=9,lastline=12]{\codedir/Semantics/Something.h}
\end{codesection}

\begin{codelisting}
	\lstinputlisting[style=C++,caption=Something.cpp Implementacija {CC,MC,CAO,MAO}]{\codedir/Semantics/Something.cpp}
\end{codelisting}

\begin{codesection}
\lstinputlisting[style=stdout,caption=Terminal]{\codedir/Semantics/terminal.std}
\end{codesection}

\pagebreak

%%%%%%%%%%%%%%%%%%%%%%%%%%%%%%%%%%%%%%%%%%%%%%%%%%%%%%%%%%%%%%%%%%%%%%%%%%%
\section{Smart Pokazivači}
\label{sec:sptr}
\setcounter{lstlisting}{0}

\emph{Smart pointer} je klasa koja upravlja dinamički alociranim objektima(resursima) i pravilno ih oslobađa(briše).\\

Tri najzastupljenija \emph{smart pointer}a su \emph{shared}, \emph{unique} i \emph{weak} pokazivači.
\begin{itemize}
	\item \emph{Shared} - održava pristup i zajedničko vlasništvo nad objektom preko pokazivača, tako da nekoliko \tottf{shared\_ptr} pokazivača mogu imati vlasništvo nad istim objektom.
Takav objekt je oslobođen tek ako su svi \tottf{shared\_ptr} uništeni, ili nakon što zadnjem preostalom \tottf{shared\_ptr} pridjelimo neki drugi objekt. To je implementirano preko brojača - \emph{reference counting}
	\item \emph{Unique} - ima jedinstveno vlasništvo nad objektom, koji se oslobađa kad se \tottf{unique\_ptr} uništi. Zbog toga što je jedinstven, primjenjuje se samo \emph{move} semantika(\textit{'krađa'})
	\item \emph{Weak} - samo pokazuje na objekt bez vlasništva nad njim, te se ne broji(\emph{reference counting}).
\end{itemize}

\begin{codelisting}
\lstinputlisting[style=C++,caption=Program.cpp,firstline=10,lastline=34]{\codedir/Smart Pointers/Program.cpp}
\end{codelisting}

\QandA%
{What would happen if you were to call function Name on w?}%
{\emph{Weak pointer} ne može pristupati member funkcijama}

\pagebreak

\QandA%
{Cast u to an r-value and use it to initialize u1. Can you use u to initialize u1 without using std::move?}
{Može postojati samo jedan \emph{unique pointer}, pa ga moramo \textit{"krasti"}.}

\QandA%
{Call the function Name on u2 and output the result. Can you call function Name on u1?}
{Ne, jer smo ukrali od \tottf{u1}, a jedinstven je samo \tottf{u2}.}


\begin{codesection}
\lstinputlisting[style=stdout,caption=Terminal]{\codedir/Smart Pointers/terminal.std}
\end{codesection}

\pagebreak

%%%%%%%%%%%%%%%%%%%%%%%%%%%%%%%%%%%%%%%%%%%%%%%%%%%%%%%%%%%%%%%%%%%%%%%%%%%
\section{Nasljeđivanje}
\label{sec:nas}
\setcounter{lstlisting}{0}

Pomoću nasljeđivanja se kreira hijerarhija zaposlenika \tottf{Employee} koja služi kao baza za klase \tottf{Worker} i \tottf{Manager}.
Unutar \tottf{Employee} imamo definiran kalup članskih funkcija pomoću \tottf{virtual} te \tottf{override} u izvedenim klasama.

\begin{codelisting}
\lstinputlisting[style=C++,caption=Employee.h,firstline=6,lastline=22]{\codedir/Generalization/Employee.h}
\end{codelisting}
\begin{codelisting}
\lstinputlisting[style=C++,caption=Worker.h,firstline=5,lastline=33]{\codedir/Generalization/Worker.h}
\end{codelisting}
\begin{codelisting}
\lstinputlisting[style=C++,caption=Manager.h,firstline=5,lastline=27]{\codedir/Generalization/Manager.h}
\end{codelisting}

\QandA%
{Add a dynamic object of type Employee and assign it to a variable named employee. Can you assign employee to manager?}
{Klasa Employee je baza izvedene klase Manager}
\begin{codesection}
\lstinputlisting[style=C++,caption=Manager.h,firstline=5,lastline=5]{\codedir/Generalization/Manager.h}
\end{codesection}

\pagebreak

\QandA%
{Delete the objects that pointers from the vector point to. What would happen if you tried to assign a wstring to a variable of type Employee?}
{Konstruktor za Employee je explicit}

\begin{codesection}
\lstinputlisting[style=C++,caption=Employee.h,firstline=10,lastline=12]{\codedir/Generalization/Employee.h}
\end{codesection}

\begin{codesection}
\lstinputlisting[style=stdout,caption=Terminal]{\codedir/Generalization/terminal.std}
\end{codesection}

\end{document}
