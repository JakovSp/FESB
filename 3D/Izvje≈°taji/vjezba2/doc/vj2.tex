\input{~/Documents/LatexPresets/programming/preambule.tex}

\newcommand{\basedir}{~/FESB/2. semestar/3D Simulacije/Izvještaji/vjezba2}
\title{Vježba L01 II}
\author{Jakov Spahija}

\begin{document}
\maketitle
\tableofcontents
\pagebreak

%%%%%%%%%%%%%%%%%%%%%%%%%%%%%%%%%%%%%%%%%%%%%%%%%%%%%%%%%%%%%%%%%%%%%%%%%%%%%%%%%%
\section{Interfaces}
\label{sec:iface}

Sučelje je apstraktna klasa samo sa čistim virtualnim funkcijama, bez ikakvog alociranog objekta. Predstavlja zahtjeve koje treba ispuniti objekt koji će implementirati korisnički kod.

\begin{codesection}
	\lstinputlisting[style=C++,firstline=5,lastline=11,caption=IAccount.h,label=lst:iacch]{\basedir/Interfaces/IAccount.h}
\end{codesection}

Svaka klasa koja implementira neko sučelje, mora zadovoljavati potpis funkcije definiranog unutar sučelja.\\
Implementacija sučelja \tottf{IAccount} unutar druge datoteke:

\begin{codelisting}
	\lstinputlisting[style=C++, caption=Account.h, label=lst:acch]{\basedir/Interfaces/Account.h}
\end{codelisting}


\pagebreak
\begin{codesection}
	\lstinputlisting[style=stdout,caption=Terminal]{\basedir/Interfaces/terminal.std}
\end{codesection}


\pagebreak

%%%%%%%%%%%%%%%%%%%%%%%%%%%%%%%%%%%%%%%%%%%%%%%%%%%%%%%%%%%%%%%%%%%%%%%%%%%%%%%%%%
\section{Operators}
\label{sec:ops}
\setcounter{lstlisting}{0}

\emph{Smart pointer}i iz prve vježbe implementiramo pomoću \emph{template} konstrukcija koja djeluje kao šablone i vežu za klasu, strukturu ili funckiju koja slijedi nakon definicije predloška.

\begin{codelisting}
	\lstinputlisting[style=C++,firstline=5,lastline=17,caption=Smart Pointer.h: Klasa,label=lst:sptrclass]{\basedir/Operators/Smart Pointer.h}
\end{codelisting}

Kod ove implementacije smart pokazivača, \emph{assignment} operator bi trebao 'ukrasti' na ono što pokazuje smart pokazivač sa desne strane jednakosti, efektivno koristeći \emph{move} semantiku.

\begin{codesection}
	\lstinputlisting[style=C++,firstline=19,lastline=28,caption={operator=()},label=lst:assignop]{\basedir/Operators/Smart Pointer.h}
\end{codesection}

\emph{Dereference} i \emph{arrow} operatori imaju svoje \tottf{\textbf{const}} verzije funkcija, kao što je navedeno u Prijepisu \reflst{ops}{sptrclass}

\begin{codesection}
	\lstinputlisting[style=C++,firstline=36,lastline=40,caption=operator*(),label=lst:defop]{\basedir/Operators/Smart Pointer.h}
\end{codesection}

\begin{codesection}
	\lstinputlisting[style=C++,firstline=48,lastline=52,caption=operator->(),label=lst:strop]{\basedir/Operators/Smart Pointer.h}
\end{codesection}

\QandA
{What would happen if you were to make another call to function m on sp?}
{Zbog definicije \emph{assignment} operatora koji \emph{move} semantiku, \tottf{sp} pokazivaču je ukraden resurs na koji je pokazivao, te je pozvan destruktor za njega.}
\begin{codesection}
	\lstinputlisting[style=C++,firstline=23,lastline=25,caption=Program.cpp]{\basedir/Operators/Program.cpp}
\end{codesection}

\begin{codesection}
	\lstinputlisting[style=stdout,caption=Terminal]{\basedir/Operators/terminal.std}
\end{codesection}

\pagebreak

%%%%%%%%%%%%%%%%%%%%%%%%%%%%%%%%%%%%%%%%%%%%%%%%%%%%%%%%%%%%%%%%%%%%%%%%%%%%%%%%%%
\section{Associations}
\label{sec:assoc}
\setcounter{lstlisting}{0}

U ovom zadatku promatramo dvije asocijacije između \tottf{Boss} i \tottf{Mafia} klasa, te \tottf{Date} i \tottf{Person} klasa.

\begin{codesection}
	\lstinputlisting[style=C++,firstline=7,lastline=10,caption=Boss.h]{\basedir/Associations/Boss.h}
\end{codesection}

\begin{codelisting}
	\lstinputlisting[style=C++,firstline=18,lastline=24,caption={Boss.h: Klasa Boss, veza na mafiju}]{\basedir/Associations/Boss.h}
\end{codelisting}

\begin{codelisting}
	\lstinputlisting[style=C++,caption=Date.h]{\basedir/Associations/Date.h}
\end{codelisting}

\begin{codelisting}
	\lstinputlisting[style=C++,caption=Person.h]{\basedir/Associations/Person.h}
\end{codelisting}

\begin{codesection}
	\lstinputlisting[style=stdout,caption=Terminal]{\basedir/Associations/terminal.std}
\end{codesection}

\pagebreak

%%%%%%%%%%%%%%%%%%%%%%%%%%%%%%%%%%%%%%%%%%%%%%%%%%%%%%%%%%%%%%%%%%%%%%%%%%%%%%%%%%
\section{Templates}
\label{sec:temps}
\setcounter{lstlisting}{0}

\emph{Template} je predložak ili šablona za klasu, strukturu ili funkciju. Može sadržavati \emph{type} parametar (\tottf{\textbf{typename}}), običan parametar ili sam template. Kompajer generira kod tijekom \emph{compile time} ovisno o tome koji su argumenti proslijeđenit tijekom korištenja koda.\\

Cilj zadataka je implementirati stog, koji može biti definiran za bilo koji tip, koristeći \tottf{\textbf{template}}.

\begin{codelisting}
	\lstinputlisting[style=C++,firstline=6,lastline=21,caption=Stack klasa]{\basedir/Templates/Stack.h}
\end{codelisting}

\begin{codelisting}
	\lstinputlisting[style=C++,firstline=39,lastline=46,caption=Stack.h: Primjer šablone za funkciju]{\basedir/Templates/Stack.h}
\end{codelisting}

\pagebreak

\emph{Variadic template} sadržava parametre koji koriste \emph{rest} operator \tottf{...}, a ako se žele proširiti u normalni oblik, \emph{rest} operator se stavi kao sufiks na identifikator.

\begin{codelisting}
	\lstinputlisting[style=C++,caption=Variadic.h]{\basedir/Templates/Variadic.h}
\end{codelisting}

\begin{codesection}
	\lstinputlisting[style=stdout,caption=Terminal]{\basedir/Templates/terminal.std}
\end{codesection}

\pagebreak

%%%%%%%%%%%%%%%%%%%%%%%%%%%%%%%%%%%%%%%%%%%%%%%%%%%%%%%%%%%%%%%%%%%%%%%%%%%%%%%%%%
\section{Variadics}
\label{sec:vars}
\setcounter{lstlisting}{0}

\begin{codelisting}
	\lstinputlisting[style=C++,caption=Templates.h]{\basedir/Variadics/Templates.h}
\end{codelisting}

\begin{codelisting}
	\lstinputlisting[style=C++,caption=Program.cpp]{\basedir/Variadics/Program.cpp}
\end{codelisting}

\begin{codesection}
	\begin{stdout}{Debug Window}{}
'Variadics.exe' (Win32): Loaded 'C:\Users\Jakov\FESB\2. semestar\3D Simulacije\Izvještaji\vjezba2\Debug\Variadics.exe'. Symbols loaded.
'Variadics.exe' (Win32): Loaded 'C:\Windows\SysWOW64\ntdll.dll'. 
'Variadics.exe' (Win32): Loaded 'C:\Windows\SysWOW64\kernel32.dll'. 
'Variadics.exe' (Win32): Loaded 'C:\Windows\SysWOW64\KernelBase.dll'. 
'Variadics.exe' (Win32): Loaded 'C:\Windows\SysWOW64\vcruntime140d.dll'. 
'Variadics.exe' (Win32): Loaded 'C:\Windows\SysWOW64\ucrtbased.dll'. 
The thread 0x34ec has exited with code 0 (0x0).
a b 
'Variadics.exe' (Win32): Loaded 'C:\Windows\SysWOW64\kernel.appcore.dll'. 
'Variadics.exe' (Win32): Loaded 'C:\Windows\SysWOW64\msvcrt.dll'. 
'Variadics.exe' (Win32): Loaded 'C:\Windows\SysWOW64\rpcrt4.dll'. 
The thread 0x3c08 has exited with code 0 (0x0).
The thread 0x29ec has exited with code 0 (0x0).
The program '[1696] Variadics.exe' has exited with code 0 (0x0).
	\end{stdout}
\end{codesection}


\pagebreak

%%%%%%%%%%%%%%%%%%%%%%%%%%%%%%%%%%%%%%%%%%%%%%%%%%%%%%%%%%%%%%%%%%%%%%%%%%%%%%%%%%
\section{Perfect Forwarding}
\label{sec:pf}
\setcounter{lstlisting}{0}

S obzirom da izrazi imaju i svoju vrijednosnu kategoriju, da bi se pravilno proslijedila kategorija preko \emph{template} argumenta, koristimo koncept \emph{universal reference}. Univerzalna referenca je parametar ili varijabla koja ima kao za svoj tip \tottf{T\&\&}.

Pravila sažimanja reference:
\begin{enumerate}
	\item \tottf{T\& \&} postane \tottf{T\&}
	\item \tottf{T\& \&\&} postane \tottf{T\&}
	\item \tottf{T\&\& \&} postane \tottf{T\&}
	\item \tottf{T\&\& \&\&} postane \tottf{T\&\&}
\end{enumerate}

\vspace{1.5em}

Pravilo kod dedukcije \emph{template} argumenata:

\begin{C++}{}{}
template<typename T>	
	void foo(T&&);
\end{C++}

\begin{enumerate}
	\item Kad se \tottf{foo} proslijedi \emph{lvalue} tipa \tottf{A}, tad se \tottf{T} zamjeni sa \tottf{A\&}, a nakon primjene pravila sažimanja postane \tottf{A\&}
	\item Kad se \tottf{foo} proslijedi \emph{rvalue} tipa \tottf{A}, tad se \tottf{T} zamjeni sa \tottf{A} a argument postane \tottf{A\&\&}
\end{enumerate}

\vspace{1.5em}

\tottf{std::forward} uspješno prosljeđuje vrijednosnu kategoriju izraza, za razliku od \tottf{std::move} koji bezuvjetno pretvara izraz iz \emph{lvalue} kategoriju u \emph{rvalue}.


\begin{codesection}
\begin{C++}{std::forward}{stdforward}
	template<class S>
	S&& forward(typename remove_reference<S>::type& a) noexcept{
		return static_cast<S&&>(a);
	}
\end{C++}
\end{codesection}

\begin{codesection}
\begin{C++}{std::move}{stdmove}
	template<class T>
	typename remove_reference<T>::type&&
	std::move(T&& a) noexcept{
		typedef typename remove_reference<T>::type&& RvalRef;
		return static_cast<RvalRef>(a);
	}
\end{C++}
\end{codesection}

\pagebreak

\begin{codelisting}
	\lstinputlisting[style=C++,firstline=7,lastline=17,caption=Functions.h: perfect forwarding]{\basedir/Perfect forwarding/Functions.h}
\end{codelisting}

\begin{codelisting}
	\lstinputlisting[style=C++,firstline=19,lastline=23,caption=Functions.h: Variadic primjer]{\basedir/Perfect forwarding/Functions.h}
\end{codelisting}


\pagebreak
\begin{codesection}
	\lstinputlisting[style=stdout,caption=Terminal]{\basedir/Perfect forwarding/terminal.std}
\end{codesection}


\end{document}
